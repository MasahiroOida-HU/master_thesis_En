\begin{abstract}

    As theoretically explained, hadrons that constitute matter acquire mass dynamically through spontaneous chiral symmetry breaking. The degree of chiral symmetry breaking is expressed by the expectation value of quark condensate (\(< q\bar{q} >\)). Since \(< q\bar{q} >\) approaches zero in ultra-high temperature and high-density regions, chiral symmetry restoration is expected. Consequently, hadron mass changes are also considered to occur. When quarks and gluons reach an ultra-high temperature and high-density state, a phase transition from the hadron phase to the quark-gluon plasma (QGP) phase occurs. QGP can be created through high-energy heavy-ion collision experiments. In other words, chiral symmetry restoration is expected within QGP and is generated by high-energy heavy-ion collisions. Light vector mesons are theoretically predicted to exhibit significant mass distribution changes. So far, searches for mass modifications of light vector mesons (\(\rho, \omega, \phi\)) inside QGP have been conducted using lepton pairs. Light vector mesons serve as probes for hadron mass within QGP. They decay exclusively into lepton pairs, which do not strongly interact with QGP. Additionally, due to their relatively short lifetimes, they tend to decay inside QGP at low transverse momentum, providing information on hadron mass within QGP.\@ However, the observed changes in the mass distribution of lepton pairs can be explained by mechanisms other than chiral symmetry restoration, and a definitive conclusion has not yet been reached.  
       
    Therefore, I aim to investigate chiral symmetry restoration by measuring muon pairs using the forward detector system of the ALICE experiment during LHC Run 3 and clarifying the transverse momentum dependence of light vector meson mass distribution changes in lead nucleus collision events. The forward region of the ALICE experiment is equipped with high-performance detectors for muon identification. Unlike electron pairs, muon pairs are not produced from \(\pi^0\) Dalitz decays and their contribution from photon (\(\gamma\)) conversions into electron pairs in materials is minimal. Thus, muon pair measurements offer a better signal-to-background ratio than electron pair measurements. In Run 3, a new silicon detector (MFT) was introduced into the ALICE forward detector system. This enables precise measurement of muon production points and improves the accuracy of pseudo-rapidity and azimuthal angle measurements. The MFT is expected to enhance the removal of heavy-flavour muons based on their different lifetimes, improve low transverse momentum measurement accuracy, and increase the mass resolution of muon pairs. 

    This master's thesis presents the transverse momentum dependence of \(\omega, \phi\) mesons using forward muon pairs in proton collision events from ALICE Run 3. Proton collision analysis serves as a reference for lead nucleus collision analysis. Additionally, it plays a crucial role in evaluating and improving the new muon track reconstruction with the MFT.\@ Track reconstruction was performed using the forward muon tracking detector system (MFT-MCH-MID), and the invariant mass distribution of muon pairs was reconstructed. A fit was applied to the extracted \(\omega, \phi\) meson peaks and their yields were calculated. The transverse momentum dependence of these yields was then determined (Figure 2). Along with this analysis, optimization of the matching \(\chi^2\) cut between MFT and MCH tracks was conducted to minimize the statistical uncertainty of \(\omega, \phi\) yields under the current track reconstruction quality. Furthermore, an analysis was performed to remove falsely matched tracks in MFT-MCH matching. By optimizing the pseudo-rapidity difference (\(\Delta \eta\)) cut between MFT and MCH tracks constituting a reconstructed track in the MFT-MCH-MID system, the \(\Delta \eta\) value that effectively removes falsely matched tracks was determined.
\end{abstract}
%Shigaki-san's comment(Log)
    %\KS{You need a proper spacing after a sentence ending with a capital letter.  Use "\textbackslash @.".} <- modified
    %\KS{Introduced NOT DURING the Run 3.} <- ... during the LHC Run3 -> from the LHC Run3
    %\KS{HF muon rejection serves mostly for S/N, not mass resolutiuon.} -> Corrected that improved eta phi resolution also improves mass resolution.
    %\KS{I think you should add the outcome of your study at the end, not just listing what you did.}tudy at the end, not just listing what you did.}